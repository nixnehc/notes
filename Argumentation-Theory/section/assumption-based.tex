\section{Interlude: Structured Argumentation}


When we want a more detailed formalisation of arguments we can turn to structured argumentation, 
in which we assume a \textit{formal language} for representing \textbf{knowledge}, 
and specifying how \textit{arguments} and \textit{counterarguments} can be constructed from that knowledge.



An argument is then said to be \textit{structured} in the sense that normally the \textit{premises} and \textit{claim} of the argument are made explicit, 
and the relationship between the premises and claim is formally defined.
% 
This means we can describe arguments and attacks is structured argumentation as follows:
\begin{itemize}[itemsep=5pt,parsep=5pt,leftmargin=3em,topsep=5pt]
    \item 
    \textit{Argument}

    \item 
    \textit{Attack}
\end{itemize}









%%=============================================================================
\clearpage
\section{Assumption--based Argumentation}


Assumption--Based Argumentation (ABA) is a form of \textbf{structured argumentation}, 
the notions of argument and attack are not primitive but are instead defined in terms of other notions, 
these notions are  
\textit{rules} (in an underlying deductive system), 
\textit{assignments} and their \textit{contraries}.


ABA is an instance of AA.



\vspace{1.5em}

\textit{deductive system}: 
$(\mathcal{L},\mathcal{R})$ where $\mathcal{L}$ is a language (a set of sentences) and $\mathcal{R}$ a set of \textit{inference rules} that induces a derivability relation $\vdash$ on $\mathcal{L}$.



$Th(T) = \{\phi \in \mathcal{L} \mid T \vdash \phi\}$


An \textit{assumption--based argumentation framework} (ABA framework) is a tuple 
\[
 \langle \mathcal{L}, \mathcal{R}, A, ^- \rangle
\] 
where 
\begin{itemize}[itemsep=5pt,parsep=5pt,leftmargin=3em,topsep=5pt]
    \item 
    $\langle \mathcal{L}, \mathcal{R} \rangle$ is a deductive system;

    \item 
    $A \subseteq \mathcal{L}$ is a non--empty finite set of \textit{assumptions}; 

    \item 
    $\bar{(\cdot)}$ is a total mapping from $A$ to $\mathcal{L}$, 
    called the \textit{contrary function}, 
    {\color{purple} $\bar{a}$} is referred to as the \textit{contrary} of $a$ for $a \in A$.
\end{itemize}



Each rule in $\mathcal{R}$ with a \textit{head} and a \textit{body}, 
where the head is a sentence in $\mathcal{L}$ and the body consists of $m \geq 0$ sentences in $\mathcal{L}$.
% 
Rules can be written in different formats, 
e.g., 
a rule with head $\sigma_0$ and body $\sigma_1, \ldots, \sigma_m$ can be written as 
\[
\sigma_0 \leftarrow \sigma_1, \ldots, \sigma_m 
\qquad \text{or} \qquad
\dfrac{\sigma_1, \dots,\sigma_m}{\sigma_0}.
\]





