\section{Some extensions}
%///////////////////////////////////////


\cite{ArgAI2009}

% /////////////////////////////////////////////////////////////////////////////
\subsection{Preference--based AF}



Several works generalizing Dung's framework to handle preferences over arguments have been proposed. 
% (Amgoud and Cayrol 1998, 2002; Amgoud and Vesic 2011, 2014; Cyras 2016; Silva, Sa ́, and Alcaˆntara 2020).



\begin{df}[Preference--based Argumentation Frameworks, PAFs]
    A \textit{preference--based argumentation framework} (PAF) is a triple {\color{purple} $(Arg,\att,>)$} such that $(Arg,\att)$ is an AF and $>$ is a strict partial order  (i.e. an irreflexive, asymmetric and transitive relation) on $Arg$, 
    called \textit{preference relation}.    
\end{df}



For any $a,b \in Arg$, 
$a > b$ means that ``$a$ is better than $b$'' for the agent. 



The preference in a PAF $(Arg,\att,>)$ working as follows: 
classical argumentation semantics are used to obtain the extensions of the underlying $(Arg,\att)$, 
and then the preference relation $>$ is used to obtain a preference relation $\sqsupseteq$ over such extensions, 
so that the \textit{best extensions} w.r.t. $\sqsupseteq$ are eventually selected.


% There have been different proposals to determine the best extensions, 
% corresponding to different criteria to define $\sqsupseteq$ as explained in the following definition.




% /////////////////////////////////////////////////////////////////////////////
\subsection{Value--based AF}


The motivation behind VAFs (value--based argumentation frameworks) is to offer an explanatory mechanism accounting for choices between distinct justifiable collections $S$ and $T$, 
which are not collectively acceptable, 
i.e., $S$ and $T$ may be admissible under Dung's semantics, 
however, $S \cup T$ fails to be. 






%///////////////////////////////////////////////
\subsection{Probabilistic AF}

Recently, 
there are has been an increasing interest in modeling uncertainty in argumentation. 
% 
This has been carried out by combining probability theory with formal argumentation.
% 
The extension of Dung's abstract argumentation framework with probability theory is called \textit{probabilistic argumentation framework}.




In general a probabilistic argumentation framework consists of probabilistic arguments and probabilistic attacks \cite{Li.Ore.Nor2011,Faz.Fle.Par2015}.
% 
As shown in \cite{Man.Bis2020}, 
however, 
an argumentation framework with probabilities on both arguments and attacks can be transformed into an equivalent one with certain attack relations (i.e. the probability of each attack is $1$).\footnote{
    The equivalent transformation between those two kinds of probabilistic AFs has linear complexity \cite{Man.Bis2020}.
}
Hence, 
for the sake of brevity, 
in the section we only focus on probabilistic argumentation frameworks where only arguments are uncertain.



% \begin{df}[PrAFs \cite{Li.Ore.Nor2012}]
%     A \textit{probabilistic argumentation framework} (PrAF) is a tuple 
%     \[
%         (Ar,\to,p,p^\to)
%     \]
%     where $(Ar,\to)$ is an AF, 
%     $p$ and $p^\to$ are, respectively, 
%     functions assigning a non-zero probability value to each argument in $Ar$ and attack in $\to$, 
%     that is,
%     $p \colon Ar \to (0,1] \cap \mathbb{Q}$ and 
%     $p^\to\!: \to \; \to (0,1] \cap \mathbb{Q}$.
% \end{df}


\begin{df}[Probabilistic Argumentation Frameworks, PrAFs]
    A \textit{probabilistic argumentation framework} (PrAF) is a tuple 
    \[
        {\color{purple} PAF = (Arg,\att,P)}
    \]
    where $(Arg,\att)$ is an argumentation framework and
    $P$ is a function assigning a non-zero probability value to each argument in $Arg$,
    that is,
    $P \colon Arg \to (0,1]$ 
    (or $P \colon Arg \to (0,1] \cap \mathbb{Q}$ to facilitate the calculation). 
\end{df}



% \begin{remark}
%     In some literature, 
%     for instance ,

%     the probability function is just $p \colon Ar \to (0,1]$.
% \end{remark}



Intuitively, 
the value assigned by $P$ to an argument $a$ represents the probability that $a$ actually occurs, 
% whereas the value assigned by $p^\to$ to a defeat $\langle a,b \rangle$ represents the conditional probability that $a$ defeats $b$ given that both $a$ and $b$ occur.
moreover, 
every attack $a \att b$ occurs with conditional probability 1, 
that is, 
$a$ attacks $b$ whenever both $a$ and $b$ occur.
% 
Or in a \textit{justification perspective}:
$P(a)$ is treated as the probability that $a$ is a justified point and therefore should appear in the graph, 
and $1-P(a)$ is the probability that $a$ is not a justified point and so should not appear in the graph.

Thus, 
an argument $a \in Arg$ is viewed as a probabilistic event which is independent from the other events associated with other arguments 
$a \not=b \in Arg$.

\vspace{1.5em}






The meaning of a PrAF is given in terms of \textit{possible worlds}. 
Given a PrAF $PAF=(Arg,\att,P)$, 
a \textit{possible world} of $PAF$ is an AF $(Arg',\att')$ which is the restriction of $(Arg,\att)$ to $Arg'$, 
that is, 
$Arg' \subseteq Arg$ and $\att' \;=\; \att \cap\; (Arg'\times Arg')$. 
% 
The set of all possible worlds of $PAF$ is denoted as {\color{purple} $pw(PAF)$}.



% \begin{df}[Subgraph]
% \label{def: subgraph}
%     For any given PrAF $\mathcal{PF}=(Ar,\to,p)$,     
%     an AF $G = (Ar',\to')$ is a \textit{subgraph} $G$ of $\mathcal{PF}$,  $G \sqsubseteq \mathcal{PF}$ in symbol, 
%     if $G$ is the restriction of the underlying AF $(Ar,\to)$ to $Ar'$, 
%     i.e., 
%     $Ar' \subseteq Ar$ and $\to' \;= (Ar'\times Ar')\; \cap \to$. 
%     % 
%     The set of all subgraphs of $\mathcal{PF}$ is denoted as $\mathsf{sub}(\mathcal{PF})$.
% \end{df}


% \begin{remark}
%     \textit{possible world}  



%     In literature, 
%     the \textit{subgraph} in Def.~\ref{def: subgraph} is also called a \textit{full subgraph} \cite{Hun2013,Dod.Wol2014}.
%     % 
%     Obviously, 
%     every $Ar' \subseteq Ar$ induces a unique (full) subgraph, 
%     and the number of subgraphs is equals to $|\wp(Ar)| - 1$.
% \end{remark}




The probability distribution over arguments can generate a probability distribution over  possible worlds (subgraphs) of the original argument framework. 
% 
Using the subgraphs, 
we can then explore the notions of probability distributions over admissible sets, extensions, and inferences.




The intuition hereby is as follows: 
for $a \in Arg$, 
$P(a)$ is the probability that $a$ belongs to an arbitrary possible world (full subgraph) of $PAF$; 
while $1-P(a)$ is the probability that $a$ does not exist in an arbitrary possible world (full subgraph). 


(using the justification perspective \cite{Hun2012}, 
where $P(a)$ means the probability that $a$ is justified in appearing in the graph) 



\vspace{1.5em}


We can use the probability assigned to each argument to generate a probability distribution over the subgraphs. 
So each subgraph can be viewed as an ``interpretation'' of what the argument graph should be. 
If all the arguments have probability $1$, 
then the argument graph itself has probability $1$, 
and that is the only interpretation we should consider (using the constellations approach). 
But, if one or more arguments has a probability less than $1$, 
then there will be multiple ``interpretations''. 
% 
So for instance, 
if there is exactly one argument $a$ in the framework $G$ with probability less than $1$, 
then there are two “interpretations”, 
the first with $a$, 
and the second without $a$. 
% 
So using the justification perspective, 
with the constellations approach, 
we can treat the set of subgraphs of a $G$ as a \textit{sample space}, 
where one of the subgraphs is the ``true'' argument graph.


\vspace{1.5em}




\begin{df}
    The \textit{interpretation} for a PrAF $PAF=(Arg,\att,P)$ is  function 
    % {\color{purple} probability distribution function} 
    $I$ over the set $pw(PAF)$ such that 
    each $w \in pw(PAF)$ is assigned by $I$ the probability 
    \[
        I(w) = \left( \prod_{a\in Arg(w)} P(a) \right) 
        \cdot
        \left( \prod_{a \in Arg \setminus Arg(w)} (1 - P(a)) \right)
    \]
    where $Arg(w)$ is the set of arguments in $w$.
\end{df}


Thus, 
the probability of a subgraph captures the degree of certainty that the subgraph contains exactly the arguments that are regarded as holding.


\begin{thm}[\cite{Li.Ore.Nor2011}, revised]
    Let $PAF=(Arg,\att,P)$ be a PrAF, 
    the function $I$ is a \textit{probability distribution} on the set $pw(PAF)$, 
    i.e., 
    $I$ is a function such that 
    $\sum_{w \in pw(PAF)} I(w) = 1$.
\end{thm}
\begin{proof}
    By induction on the size of $Arg$. 
    See the proof of \cite[Prop.~3 in p.~60]{Hun2013} for details.
\end{proof}




\begin{example}
    Considering following PrAF 
    \[G: \qquad 
    \begin{tikzcd}
        a_1/ 1 \arrow[r, bend left=35] & a_2 / 1 \arrow[l, bend left=35] & a_2 / 0.8 \arrow[l]
    \end{tikzcd}\]
    % 
    Clearly $G$ has eight subgraphs, 
    then we get the following probability distribution over the subgraphs:
    % 
    \[
      \renewcommand{\arraystretch}{1.3} % 行距
      \renewcommand{\arraycolsep}{1.2em} % 列距
    \begin{array}{ccccccccc}
        \hline 
        & G_1 = G & G_2 & G_3 & G_4 & G_5 & G_6 & G_7 & G_8 \\ 
        
        I 
        & 0.8 & 0.2 & 0  \\
        \hline
    \end{array}
    \]
\end{example}




If all the arguments in a PrAF $G$ have probability of $1$, 
then the only subgraph of $G$ to have non-zero probability is itself, 
and so it has probability $1$. 
% 
At the other extreme,
if all the arguments in a probabilistic argument graph $G$ have probability of $0$, 
then the empty graph has probability $1$.








\vspace{3em}


Given a PrAF  $G = (Arg,\att,P)$, 
and a set of arguments $\Gamma \subseteq Arg$, 
we want to calculate the probability that $\Gamma$ is a $\sigma$--extension, 
which we denote by {\color{purple} $P^\sigma_G(\Gamma)$},
where $\sigma \in \{ad, co, pr, gr, st\}$. 
For this, we take the sum of the probability of the full subgraphs for which $\Gamma$ is a $\sigma$--extension.



The analogous {\color{teal} \textit{credulous acceptance problem}} in the context of a probabilistic argumentation framework, 
i.e. the probability that a given set of arguments is an extension under a given semantics, is the following.

% \[
%     P_{PAF}^{\sigma,\circ} (a)
% \]





\begin{df}[Probabilistic credulous acceptance]
    Given a PrAF $PAF=(Arg,\att,P)$, 
    an argument $a \in Arg$, 
    the probability $PrCA^\sigma_{PAF}(a)$ that $a$ is credulously acceptable w.r.t. semantics $\sigma$ is 
    \[
        PrCA^\sigma_{PAF} (g)
            = 
        \sum_{w \;\in\; pw(PAF) ~\land~ a\; \in\; \bigcup \sigma(w)} I(w).
    \]
\end{df}


Thus, 
the probability that an argument $a$ is credulously accepted according to a semantics $\sigma$ is defined as the sum of the probabilities of the possible worlds  of a PrAF for which argument $a$ is credulously accepted.





Computing $PrCA^\sigma_{PAF}(a)$ is $FP^{\#P}$--hard for $\sigma \in \{\mathcal{GR,PR,ST},\mathcal{STT}\}$ [Fazzinga, Flesca, and Furfaro 2018], 
where $FP^{\#P}$ is the class of functions computable by a polynomial--time Turing machine with a $\#P$ oracle.





Probabilistic credulous acceptance does not express the probability that a given argument is accepted. 
% 
However, 
a new problem, called \textit{Probabilistic Acceptance}, 
was given in \cite{Alf.Cal.Gre.Par.Tru2020} can be intuitively stated as follows.
% 
Given a probabilistic AF, 
a semantics $\sigma$ and a goal argument $a$, 
computing the probability that $a$ is accepted. 



\begin{df}[Probabilistic Acceptance]
    Given a PrAF $PAF=(Arg,\att,P)$ and an argument $a \in Arg$, 
    the probability $PrA^\sigma_{PAF}(a)$ that $a$ is acceptable w.r.t. semantics $\sigma$ is
    \[
        PrA^\sigma_{PAF}(a)
        =
        \sum_{w \in pw(PAF) \land E \in \sigma(w) \land g \in E} I(w) \cdot Pr(E,w,\sigma)
    \]
    where $Pr(\cdot,w,\sigma)$ is a $PDF$ over the set $\sigma(w)$.
\end{df}







% \subsubsection{the shortcomings of Independence Assumption}


% Whilst assuming independence brings some advantages, 
% there are situations where it is not appropriate as it does not capture the uncertainty correctly. 
% To illustrate this, 
% we consider the following example.



% \vspace{3em}

% \dotfill




% epistemic approach to probabilistic argumentation [Thimm, 2012; Hunter and Thimm, 2014; Baroni et al., 2014; Hunter and Thimm, 2017].





% % /////////////////////////////////////
% \subsubsection{Other approaches in probabilistic argumentation}


% combining the epistemic and the constellation approach

% apply probabilistic techniques in order to learn argumentation frameworks from data

% model changing beliefs in argumentation

