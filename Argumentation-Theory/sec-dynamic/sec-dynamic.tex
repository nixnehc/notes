
% /////////////////////////////////////////////////////////////////////////////
\section{Dynamic}

An important feature of the argumentation forms is that, in practice, these are not \textit{static} systems:
% 
a particular AF $\mathcal{AF}=(Ar,\to)$ 
represents only a  ``snapshot'' of the environment, 
and, as further facts, 
information and opinions emerge the form of the initial view may change significantly in order to accommodate these. 
For example, 
additional arguments may have to be considered so changing $\mathcal{AF}$; 
existing attacks may cease to apply and new attacks (arising from changes to $\mathcal{AF}$) come into force.


It is clear that accounting for such dynamic aspects raises a number of issues in terms of assessing the acceptability status of individual arguments.




新的论证产生, 由此又得到新的攻击关系。

由于得到了新的知识,两个论证之间增加了攻击关系。


\vspace{2em}

抽象论证框架的变化主要表现为论证以及论证之间攻击关系的增加或减少。

仅增加论证或攻击关系的论辩框架变化被称为\textit{抽象论辩框架的扩张}(expansion);

仅减少论证或攻击关系的论辩框架变化被称为\textit{抽象论辩框架的限制}(abstraction ?)。

\vspace{1em}



强扩张(strong expansion):增加的论证不能被原论辩框架中的论证攻击。

弱扩张(weak expansion):增加的论证不能攻击原论辩框架中的论证。

标准扩张(normal expansion):不对此做任何要求。



\vspace{3em}

抽象论辩系统的动态性研究抽象论辩框架和论辩语义的变化。

