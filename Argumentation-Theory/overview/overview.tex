\section{Overview}



% /////////////////////////////////////////////////////////////////////////////
\subsection{Belief Revision and Argumentation}

There are several works that combine belief revision and argumentation. 
% 
Coste-Marquis et al. \cite{Cos.Kon.Mai.Mar2015} introduce novel enforcement strategies for modifying argumentation frameworks to make sure that a specific set of arguments is part of its extension, which are close to the process of belief revision. 
% 
Baroni et al. compare and relate belief revision and argumentation as approaches to model reasoning processes \cite{Bar.Fer.Gia.Sim2022}. 
% 
Coste-Marquis et al. [On the revision of argumentation systems: minimal change of arguments statuses, in: Proceedings of the
Fourteenth International Conference on Principles of Knowledge Representation and Reasoning, 2014, pp. 52–61.] derive argumentation systems that satisfy given revision formulas, i.e., given an argumentation system and a revision formula that expresses how the statuses of some arguments have to be changed under a chosen semantics, the derived argumentation systems are such that the corresponding extensions are as close as possible to the extensions of the input system.
% 
% 
Diller et al. \cite{Dil.Har.Lin.Rum.Wol2018} follow this work and study how to update an argumentation framework with new information, 
either a formula or another argumentation framework, 
and defines rationality criteria for such updates. 
% 
% 
Paglieri and Castelfranchi propose a data-oriented belief revision that enables incorporating computational argumentation \cite{Pag.Cas2005}. 
% 
% 
Booth et al. \cite{Boo.Kac.Rie.van2013} propose two methods to restore consistency when an agent's belief state, 
composed of a Dung's argumentation framework and a propositional constraint, 
are inconsistent: 
firstly, a normal expansion that may alter the set of complete labellings; 
and secondly, belief revision techniques that aim to retain similarity to the original labellings.






% /////////////////////////////////////////////////////////////////////////////
% \subsection{Probability and Argumentation}