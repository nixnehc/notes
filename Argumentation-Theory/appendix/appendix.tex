\section*{Appendix}
\appendix

\section{Probability Measures and Distributions}



Let $S$ be a nonempty set and $\mathcal{A}$ an algebra of subsets of $S$, 
i.e., a set of subseteq of $S$ s.t. (i) $S \in \mathcal{A}$, 
and (ii) if $H_1,H_2 \in \mathcal{A}$ then $\overline{H}_1, H_1 \cup H_2 \in \mathcal{A}$, 
where $\overline{H}_1=S\setminus H_1$.



A \textit{(finitely additive) probability measure} is a function $p \colon \mathcal{A} \to [0,1]$ such that:
\begin{enumerate}[itemsep=5pt,parsep=5pt,leftmargin=3em,topsep=5pt,label=(\arabic*)] %% or label = \alph*, \roman*
    \item $p(S)=1$,
    
    \item $p(H_1 \cup H_2) = p(H_1)+p(H_2)$, 
    whenever $H_1 \cap H_2 = \emptyset$.
\end{enumerate}


The triple $(S,\mathcal{A},p)$ is called a \textit{probability space}, 
and elements of $\mathcal{A}$ are called \textit{measurable sets}.



For any probability measure $p$ and $H,A \in \mathcal{A}$ such that $p(H)>0$, 
the \textit{conditional probability} $p(A | H)$ is defined in the usual way, that is, 
\[
    p(A \mid H) = 
    \dfrac{p(A \cap H)}{p(H)}. 
    % \qquad (p(H)>0)
\]
It is known that the function $p(\cdot | H)$ is also a probability measure.


\framebox[15cm]{
    Bayes' Theorem: $p(A|B) = \dfrac{p(B|A) \cdot p(A)}{p(B)}$
}



For a finite set $S$, 
a \textit{probability distribution} is any  function $d\colon S \to [0,1]$ such that 
\[
    \sum_{s \in S} d(s) = 1.
\]
Every probability distribution $d$ on a finite set $S$ induces a function $p$ on the $\wp(S)$:
\[
    p(H) = \sum_{s \in H} d(s),
\]
which is a (finitely additive) probability measure. 
% 
On the other hand,
any probability measure $p$ on the power set of $S$ induces a distribution $d(s)=p(\{s\})$, 
thus, 
on finite sets ``measures = distributions''.



% \begin{df}dd
    
% \end{df}