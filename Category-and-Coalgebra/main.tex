\documentclass[12pt]{article} % Document class article only supports 10pt, 11pt, and 12pt.
\usepackage[utf8]{inputenc}
\usepackage[english]{babel}
\usepackage{cx}


%% 设置页边距:
\usepackage{geometry}
\geometry{left=3cm,right=3cm,top=2cm,bottom=2cm}

\usepackage{indentfirst}

%% 行距、段距设置
%\linespread{1.2}     % 设置基本行距。
% article 文档类的默认是1,即1.2倍字号大小;
% ctexart 文档类的默认是1.3,即1.56倍字号大小。
\setlength\parskip{5pt}   % 段间距

\usepackage[
	backend=bibtex,
	style=numeric-comp,
	sorting=nyt,
	date=year,
	backref=true,
	backrefstyle=three]
	{biblatex}
\addbibresource{/Users/chenxin/Desktop/chenxin-biblatex.bib}






%%=============================================================================
\title{Notes on Category \& Coalgebra}
\author{Xin Chen \qquad \href{mailto:chenxin_hello@outlook.com}{\textsf{chenxin\_hello@outlook.com}}  
	\qquad 
	$Q_{uality} = \int (\text{\Large $K$},P,t)$}
\date{latest update: \today}

%%-------------------------------------
\begin{document}

\maketitle
\tableofcontents


\medskip

citation testing: \cite{Smi2024,Lan2018,SEP-category}



\section{Basic Category Theory}


%%-------------------------------------
\subsection{Categories}


\begin{df}[Categories]
	A \textbf{category} $\mathsf{C}$ consists of a class {\color{purple} $\mathsf{ob(C)}$} of \textbf{objects}, 
	and for each pair of objects $A,B$, 
	a family {\color{purple} $\mathsf{C}(A,B)$} of \textbf{arrows}.
	% 
	If $f$ is a arrow in $\mathsf{C}(A,B)$, 
	we write {\color{red} $f \colon A \to B$} or {\color{red} $A \xrightarrow{f} B$}, 
	and call $A$ the \textbf{domain} (source) and $B$ the \textbf{codomain} (target) of the arrow.
	% 
	The collection of arrows is endowed with some algebraic structure:
    \begin{enumerate}[label=(\arabic*)]
        \item \textit{Identity Arrows}: for every object $A$ of $\mathsf{C}$ there is an arrow ${\color{purple} Id_A} \colon A \to A$ (or ${\color{purple} 1_A} \colon A \to A$).
        
        \item \textit{Composition}: every pair of arrows $f \colon A \to B$, 
        $g \colon B \to C$ can be uniquely composed to an arrow $g \circ f \colon A \to C$.

        \item \textit{Associativity}: The operations are supposed to satisfy the \textit{associative law} for composition, 
        while the appropriate identity arrows are left and right neutral elements.
    \end{enumerate}

	An arrow $f \colon A \to B$ is an \textbf{iso} if it has an \textbf{inverse}, 
	that is, an arrow $g \colon B \to A$ such that 
	$f \circ g =Id_B$ and $g \circ f = Id_A$.
\end{df}



\begin{example}[Some categories]
	\begin{enumerate}
		\item For every algebraic similarity $\Sigma$, 
		the class $\mathsf{Alg}(\Sigma)$ of $\Sigma$-algebras with homomorphisms as arrows.
	\end{enumerate}
\end{example}





\begin{example}[The category of sets, $\mathsf{Set}$] 
	\zh{集合范畴}

    {\color{purple} $\mathsf{Set}$} is the category with (i) objects: all sets, and (ii) arrows: for any sets $X,Y$, every (total) function $f \colon X \to Y$ is an arrow.

    Remarks on the category $\mathsf{Set}$:
    \begin{enumerate}[itemsep=1pt,parsep=5pt,leftmargin=4em,topsep=1pt,label=(\arabic*)] 
        \item The identity function on any set is the identity arrow.

        \item The set--function $f \colon A \to B$ and $g \colon B \to C$ always compose.
        
        \item The arrows in $\mathsf{Set}$, like any category arrows, must come with determinate targets and codomains. 
        Hence, we can't simply identify an arrow in $\mathsf{Set}$ with a function's {\color{teal} graph} (i.e. the set $\hat{f}=\{\langle x,y \rangle \mid f(x)=y\}$). 

        We can define a set--function $f \colon A \to B$ as a triple $(A,\hat{f},B)$.


        \item {\color{teal} Empty set}: 
            \begin{itemize}
                \item There is an identity arrow for $\emptyset$ in $\mathsf{Set}$. 
                % 
                Vacuously, for any set $Y$ there is exactly one function $f \colon \emptyset \to Y$, that is, the one whose graph is the empty set. Hence in  particular there is a function $1_\emptyset \colon \emptyset \to \emptyset$.


                \item In $\mathsf{Set}$, 
                the empty set is the one and only set s.t. there is exactly one arrow \textit{from} it to any other set.

                (This tells us how we can characterize a significant object in a category by what arrows it has to and from other objects. 
                % 
                For example, we can define {\color{teal} \textit{singletons}} in $\mathsf{Set}$ by relying on the observation that there is exactly one arrow from any set \textit{to} a singleton.)
            \end{itemize}
        

        
    \end{enumerate}
\end{example}


\begin{example}[Logical example]
	The \textbf{formulas} and \textbf{proofs} in logic form a category.
	% 
	Formulas are objects, 
	and for any formulas $\varphi,\psi$, 
	the proofs for implication $\varphi \to \psi$, 
	we write $\vdash \varphi \to \psi$ , as arrows.

	For formula $\varphi$, 
	$id_\varphi$ is $\vdash \varphi \to \varphi$. 
	The composition of proof is just HS (hypothetical syllogism) rule:
	$ \vdash \varphi \to \psi 
		\quad +\quad 
		\vdash  \psi \to \chi 
		\quad
		\Rightarrow \quad \vdash \varphi \to \chi.
	$ 
\end{example}


The categories whose objects are sets, 
perhaps equipped with some structure (e.g. groups, monoids, etc.), 
and whose arrows are structure--respecting set--functions, 
are often called \textit{concrete categories}.  \zh{具体范畴}
\label{def: concrete-categories}


\begin{example}
    Some examples of concrete categories:
    
    \begin{itemize}[itemsep=5pt,parsep=5pt,leftmargin=3em,topsep=5pt]
        \item 
        $\mathsf{FinSet}$: the category whose objects are finite sets and whose arrows are the set--functions between such objects.
        

        \item $\mathsf{Rel}$: the category of relations.
        
        objects: all sets 

        arrows: any relation $R$ between $A$ and $B$. 
        We can let a relation $R$ be a triple $R = (A,\hat{R},B)$ where $\hat{R} \subseteq A \times B$ is $R's$ extension. 
        We allow the case where $\hat{R}$ is empty and the arrow of the form $(A,\emptyset,B)$.

        The identity arrow on $A$ is the diagonal relation whose extension is $\{(a,a) \mid a \in A\}$.
    \end{itemize}
\end{example}





\begin{thm}
	Identity arrows on a given object are unique; and the identity arrows on distinct objects are distinct.
\end{thm}



%%-------------------------------------
\subsection{Commutative diagrams}

\zh{「交换」意指箭头的合成殊途同归}

[Arbib and Manes, 1975, p. 2]: 
``\textit{commutare} is the Latin for \textit{exchange}, 
and we say that a diagram commutes if we can exchange paths, 
between two given points, with impunity.''



a diagram commutes if each minimal polygon in the diagram commutes.





%%-------------------------------------
\subsection{Constructing categories}

a number of general constructions which give us new categories from old.



% -----------------
\subsubsection{Duality or Opposite}

Maybe the simplest way of getting a new category from old is by simply \textit{reversing all the arrows}. 

\begin{df}[Opposite category]
	The \textbf{dual} or \textsf{opposite category} {\color{red} $\mathsf{C}^{op}$} of a given category $\mathsf{C}$ has the same objects as $\mathsf{C}$, 
	while $\mathsf{C}^{op}(A,B) = \mathsf{C}(B,A)$ for all objects $A,B$.
	%
	The identity arrows remain the same, 
	that is, 
	$Id^{op}_A = Id_A$.
	% 
	Composition in $\mathsf{C}^{op}$ is defines in terms of composition in $\mathsf{C}$: 
	$f \circ^{op} g = g \circ f$. 
\end{df}


% figure
\begin{figure}[h]
	\centering
	\begin{tikzcd}
		A \arrow[r, "f"] \arrow[rd, "g \circ f"'] & B \arrow[d, "g"] & \leadsto  & A & B \arrow[l,"f"'] \\
			 & C  &  &   & C \arrow[u, "g"'] \arrow[lu, "f \circ^{op} g"] \\
		\end{tikzcd}
		
		$\mathsf{C}$ and $\mathsf{C}^{op}$
	\caption{Dual category}
	\label{fig: dual-category}
\end{figure}


Clearly, 
$(\mathsf{C}^{op})^{op} = \mathsf{C}$ which means that \textit{every category is the opposite of some other category}.


\vspace{1.5em}


Let {\color{purple} $L^{cat}$} be the elementary language of categories, 
that is, 
a {\color{teal} two-sorted first-order language} with identity with one sort of variable for objects, $A,B,C,\dots$, 
and another sort for arrows $f,g,h,\dots$. 
% 
It has function symbols `$src$' and `$tar$', 
denoting two operations taking arrows to objects, 
a 2--ary function `$\circ$', 
and a relation `\dots is the identity arrow for object \dots'. 


\begin{df}[Dual formulas]
	Suppose that $\phi \in L^{cat}$, 
	its \textit{dual} {\color{purple} $\phi^{op}$} is a formula getting by (i) swapping `$src$' and `$tar$' and (ii) reversing the order of composition, i.e., 
	`$f \circ g$' becomes `$g \circ f$', 
	and so on and so forth.
\end{df}



\begin{thm}[Duality principle]
	Suppose that $\phi$ is an $L^{cat}$--sentence (without free variables), 
	that is, 
	$\phi$ is a general clain about objects/arrows in an arbitrary category. 
	% 
	Then, 
	if the axioms of category theory entail $\phi$,
	they also entail the dual claim $\phi^{op}$.
\end{thm}


The duality principle might be very simple but it is a hugely labour-saving result.








%%-----------------
\subsubsection{Subcategories}

The most usual way of getting a new category is by slimming down an old one while retaining enough categorial structure:

\begin{df}[Subcategories]
	Let $\mathsf{C}$ be a category. 
	A \textit{subcategory} $\mathsf{S}$ of $\mathsf{C}$ consists of a subclass $\mathsf{ob(S)}$ of $\mathsf{ob(C)}$ together with, 
	for each $\mathsf{S}$--object $A,B$, 
	a subclass $\mathsf{S}(A,B)$ of $\mathsf{C}(A,B)$, 
	such that $\mathsf{S}$ is closed under composition and identities. 
	% 
	It is a \textit{full subcategory} if $\mathsf{S}(A,B) = \mathsf{C}(A,B)$ for all $\mathsf{S}$--object $A,B$.
\end{df}


A full subcategory therefore consists of a selection of the objects, 
with all the maps between them. 



%%-----------------
\subsubsection{Product categories}

\begin{df}{Product categories}
	If $\mathsf{C}$ and $\mathsf{D}$ are categories, 
	a \textit{product category}	is $\mathsf{C} \times \mathsf{D}$ such that:
	\begin{enumerate}[itemsep=5pt,parsep=5pt,leftmargin=3em,topsep=5pt,label=(\arabic*)]
		\item 
		Its objects are paris $\langle C,D \rangle$ where $C$ is a $\mathsf{C}$--object and $D$ is a $\mathsf{D}$--object;
		
		\item
		Its arrows from $\langle C,D \rangle$ to $\langle C',D' \rangle$ are all the pairs $\langle f,g \rangle$ where $f \colon C \to C'$ is a $\mathsf{C}$--arrow and $g \colon D \to D'$ is a $\mathsf{D}$--arrow;

		\item 
		The identity arrow of $\langle C,D \rangle$ is $Id_{\langle C,D \rangle} = \langle Id_C, Id_D \rangle$;

		\item 
		Composition is defined componentwise: 
		$\langle f,g \rangle \circ \langle f', g' \rangle = \langle f\circ f', g \circ g' \rangle$.
	\end{enumerate}
\end{df}




%%-------------------------------------
\subsubsection{Slice categories}


% -----------------
\subsubsection{Arrow categories}


\begin{df}[Arrow categories]
	Let $\mathsf{C}$ be a category, 
	the \textit{arrow category}	 {\color{purple} $\mathsf{C}^\to$} of  $\mathsf{C}$ has: 
	\begin{enumerate}[itemsep=5pt,parsep=5pt,leftmargin=3em,topsep=5pt,label=(\arabic*)] %% or label = \alph*, \roman*
		\item 
		The $\mathsf{C}^\to$--objects are all the $\mathsf{C}$--arrow.

		\item 
		The $\mathsf{C}^\to$--arrows from $f \colon X \to Y$ to $g \colon W \to Z$ are the commutative squares in $\mathsf{C}$ formed by arrow $j \colon X \to W$ and $k \colon Y \to Z$ s.t. $k \circ f = g \circ j$.

		\begin{center}
		\begin{tikzcd}
			X \arrow[r,"j"] \arrow[d, "f"'] & W \arrow[d, "g"] \\
			Y \arrow[r, "k"']  & Z               
		\end{tikzcd}
		\end{center}

		\item 
		Composition is defined by amalgamating commuting squares in $\mathsf{C}$ to get another commuting square.

		\item 
		The identity arrow on a $\mathsf{C}^\to$--object is defined in the obvious way.
	\end{enumerate}
\end{df}




\begin{example}
	\qquad

	\begin{enumerate}[itemsep=5pt,parsep=5pt,leftmargin=3em,topsep=5pt,label=(\alph*)] 
		\item 
		{\color{purple} $\mathsf{Set}^\to$}: whose objects are set functions, 
		and whose arrows are suitable commutative squares.
	\end{enumerate}
\end{example}






%%=============================================================================
\subsection[So many arrows]{So many arrows}


Characterizing a number of different kinds of arrows by the way they interact with other arrows.


%%-------------------------------------
\subsubsection{Monomorphisms \& Epimorphisms}


\paragraph{Monomorphisms}


\begin{df}[Monomorphisms]
	An arrow $f$ in a category $\mathsf{C}$ is a \textit{monomorphism} (\textit{monic}, for short) iff it is {\color{teal} left-cancellable}, 
	that is, 
	whenever $g$ and $h$ are such that $f \circ g = f \circ h$, 
	then $g = h$.  \zh{单态射}
\end{df}


That is, 
if the composites $f \circ g$ and $f \circ h$ are to exist and be equal, 
then $g$ and $h$ must be parallel arrows sharing the same source and target. 
% 
In other words, 
if the following diagram commutes
% 
\begin{center}
\begin{tikzcd}
	A \arrow[r, "g", bend left] \arrow[r, "h"', bend right] & X \arrow[r, "f"] & Y
\end{tikzcd}
\end{center}
% 
then $g=h$.



\begin{prop}
	In the category $\mathsf{Set}$  where the arrows are set--functions, 
	$f$ is \textit{injective} as a function 
	iff
	$f$ is a \textit{monomorphism}.
\end{prop}
\begin{proof}
	\qquad

	$\Rightarrow$ \quad
	% 
	Assume that $f \colon C \to D$ is injective, 
	and for any functions $g \colon A \to C$ and $h \colon A \to C$ we have $f(g(x)) = f(h(x))$ where $x$ is arbitrary. 
	% 
	But that implies $g(x) = h(x)$, 
	thus in arrow--speak, 
	$f \circ g = f \circ h$ implies $g = h$, 
	therefore $f$ is monomorphism. 


	$\Leftarrow$ \quad
	% 
	Suppose $f \colon C \to D$ is \textit{not} injective, 
	then for some $x,y \in C$ we have that $f(x)=f(y)$ but $x \not=y$.
	% 
	Let $1$ be any singleton, 
	then $x$ and $y$ will be picked out by the functions $\bar{x} \colon 1 \to C$ and $\bar{y} \colon 1 \to C$ respectively. 
	% 
	Hence in $\mathsf{Set}$ we have $f \circ \bar{x} = f(x) = f(y) = f \circ \bar{y}$ but not $\bar{x} = \bar{y}$, 
	which means that $f$ is not left--cancellable.
\end{proof}




Thus, 
in $\mathsf{Set}$, 
the monomorphisms are exactly the injective functions.




\paragraph{Epimorphisms}

Let's see the obvious dual notion for monomorphism.

\begin{df}[Epimorphisms]
	An arrow $f$ in a category $\mathsf{C}$ is an \textit{epimorphism} (\textit{epic}, for short) iff it is {\color{teal} right--cancellable}, 
	in other words, 
	whenever $g$ and $h$ are s.t. $g \circ f=h \circ f$, 
	then $g = h$.
\end{df}


Left and right cancellability are evidently dual properties, 
$f$ is right--cancellable in $\mathsf{C}$ iff it is left--cancellable in $\mathsf{C}^{op}$. 


\begin{center}
	Left--cancellable:
	\begin{tikzcd}
		A \arrow[r, "g", bend left] \arrow[r, "h"', bend right] & X \arrow[r, "f"] & Y
	\end{tikzcd}
	
	\vspace{1em}
	\hspace{6em}$\Updownarrow$ dual
	\vspace{1em}

	Right--cancellable:
	\begin{tikzcd}
		A & X \arrow[l, "g"', bend right] \arrow[l, "h", bend left] & Y \arrow[l, "f"']
	\end{tikzcd}
	
\end{center}



\begin{prop}
	In $\mathsf{Set}$, 
	$f$ is \textit{surjective} as a function iff $f$ is an \textit{epimorphism}.
\end{prop}


\vspace{2em}

\tcbset{colback=white, width=\textwidth,boxrule=0.2mm} % boxrule调整边框线条粗细
\begin{tcolorbox}[arc=2mm,title={monic v.s epic}] % arc调整表框四角弧度;

	`\textit{mono}' means one, 
	and the `\textit{monomorphisms}' are rather often the injective, one-to-one functions. 
	While `\textit{epi}' is Greek for `\textit{on}' or `\textit{over}', 
	and the `\textit{epimorphisms}' are  fairly often surjective, onto, functions. 

\end{tcolorbox}


\vspace{2em}


\begin{thm}
	\begin{enumerate}[itemsep=5pt,parsep=5pt,leftmargin=3em,topsep=5pt,label=(\arabic*)] 
		\item 
		Identity arrows are always monic. 
		Dually, 
		they are always epic too.

		\item 
		If $f,g$ are monic, so is $f \circ g$. 
		If $f,g$ are epic, so is $f \circ g$.

		\item 
		If $f \circ g$ is monic, so is $g$. 
		If $f \circ g$ is epic, so is $f$.
	\end{enumerate}
\end{thm}






\paragraph{Symbols} 

There is a notational convention that we use special styles of drawn arrows to represent cancellable arrows:

\begin{center}
{\color{purple} $f \colon C \monic D$} or {\color{purple} $C \stackrel{f}{\monic} D$} represents a monic, left--cancellable $f$; 


{\color{purple} $f \colon C \epic D$} or {\color{purple} $C \stackrel{f}{\epic} D$} represents an epic, right--cancellable $f$.
\end{center}



The convention is easy  to remember: 
(i) a left-cancellable arrow gets notated by an extra decoration on the tail of the arrow (i.e. on the left), 
and  (ii) a right-cancellable arrow gets an extra decoration on the head (i.e. on the right).




%%-------------------------------------
\subsubsection{Inverses arrows ???}


\begin{df}[Inverse]
	Given an arrow $f \colon C \to D$ in a category $\mathsf{C}$:
	\begin{enumerate}[itemsep=5pt,parsep=5pt,leftmargin=3em,topsep=5pt,label=(\arabic*)] %% or label = \alph*, \roman*
		\item 
		$g \colon D \to C$ is a \textit{right inverse} of $f$ iff $f \circ g = Id_D$.

		\item 
		$g \colon D \to C$ is a \textit{left inverse} of $f$ iff $g \circ f = Id_C$.

		\item 
		$g \colon D \to C$ is a \textit{inverse} of $f$ iff it is both a right inverse and a left inverse of $f$.
	\end{enumerate}	
\end{df}


\begin{center}
\begin{tikzcd}
	C \arrow[r, "f", bend left] \arrow["Id_C"', loop, distance=2em, in=215, out=145] & D \arrow[l, "h", bend left] \arrow["Id_D"', loop, distance=2em, in=35, out=325]
\end{tikzcd}
\end{center}


\begin{remark}
	\begin{enumerate}[itemsep=5pt,parsep=5pt,leftmargin=2em,topsep=5pt,label=(\alph*)] %% or label = \alph*, \roman*
		\item 
		$g \circ f = Id_C$ in $\mathsf{C}$ iff $f \circ^{op} g = Id_C$ in $\mathsf{C}^{op}$.
		% 
		A left inverse in $\mathsf{C}$ is a right inverse in $\mathsf{C}^{op}$, and vice versa. 

		\item 
		The notions of a right inverse and left inverse are dual to each other, 
		and the notion of an inverse is its own dual.

		\item 
		If $f$ has a right (left) inverse $g$, 
		then $f$ is a left (right) inverse of $g$.
	\end{enumerate}
\end{remark}


\begin{thm}
	In a category where arrows are functions, 
	if $f$ has a left--inverse as an arrow, it is \textit{injective} as a function. 
	And if $f$ has a right--inverse, it is \textit{surjective} as a function.
\end{thm}


For those typical concrete categories (arrows are functions): 

\begin{center}
\begin{tabular}{lllll}
	$f$ has a left inverse & 
	$\Rightarrow$ & 
	$f$ is injective & 
	$\Rightarrow$ & 
	$f$ is monic (left--cancellable).  \\

	$f$ has a right inverse & 
	$\Rightarrow$ & 
	$f$ is surjective & 
	$\Rightarrow$ & 
	$f$ is epic (right--cancellable).  \\	
\end{tabular}
\end{center}






every epic is a left inverse in $\mathsf{Set}$ is equivalent to the {\color{teal} Axiom of Choice}.





\begin{thm}
	If an arrow has both a right inverse and a left inverse, 
	then these are the same and are the arrow's \textit{unique} inverse.
\end{thm}



%%-------------------------------------
\subsubsection{Isomorphisms}



\begin{df}[Isomorphisms]
	An \textit{isomorphism} \zh{同构} in a category $\mathsf{C}$ is an arrow which has an inverse.  
	% 
	We conventionally represent isomorphisms by decorated arrows, 
	thus: {\color{purple} $\isoarrow$}.
\end{df}




\begin{thm}
	\begin{enumerate}[itemsep=5pt,parsep=5pt,leftmargin=3em,topsep=5pt,label=(\arabic*)] %% or label = \alph*, \roman*
		\item 
		Identity arrows are isomorphisms. 

		\item 
		The (unique) inverse $f^{-1}$ of an isomorphism $f$ is also an isomorphism.

		\item 
		If $f$ and $g$ are isomorphisms, 
		then $g \circ f$ is an isomorphism if it exists, 
		whose inverse will be $f^{-1} \circ g^{-1}$.
	\end{enumerate}
\end{thm}



\begin{example}
	\qquad

	\begin{enumerate}[itemsep=5pt,parsep=5pt,leftmargin=3em,topsep=5pt,label=(\arabic*)] 
		\item 
		In $\mathsf{Set}$, the isomorphisms are the \textit{bijective} set--functions.

		\item 
		In $\mathsf{Grp}$, the isomorphisms are the \textit{bijective} group homomorphisms.
	\end{enumerate}
\end{example}



Isomorphisms are monic and epic, 
but arrows which are both monic and epic need not have inverses so need not be isomorphisms, e.g. in $\mathsf{Pos}$ and $\mathsf{Mon}$. 
However, we do have this result:


\begin{thm}
	If \textit{f} is both monic and has a right inverse (or both epic and has a left inverse), 
	then \textit{f} is an isomorphism. 
	% 
	Equivalently: 
	if $f$ is both monic and split epic (or both epic and split monic), 
	then $f$ is an isomorphism.
\end{thm}




\begin{df}
	A category $\mathsf{C}$ is \textit{balanced} iff every arrow which is both monic and epic is an isomorphism. 
	[$\monic + \epic ~\leadsto~ \monoepiarrow \quad \Rightarrow \quad \isoarrow$]
\end{df}






%%-------------------------------------
\subsubsection{Isomorphic objects}


\begin{df}
	If there is an isomorphism $f \colon C \isoarrow D$ in $\mathsf{C}$ then the object $C$ and $D$ are said to be \textit{isomorphic} in $\mathsf{C}$, 
	denoted as $C \cong D$.
\end{df}


\begin{prop}
	Isomorphism between objects in a category $\mathsf{C}$ is an equivalence relation.
\end{prop}




\begin{example}
	\begin{enumerate}[itemsep=5pt,parsep=5pt,leftmargin=3em,topsep=5pt,label=(\alph*)] %% or label = \alph*, \roman*
		\item 
		In $\mathsf{Grp}$, 
		any two Klein four--groups are isomorphic in the categorial sense.

		\item 
		In $\mathsf{Set}$, 
		any two \textit{singletons} are isomorphic. 
		% 
		More generally, 
		any two objects in $\mathsf{Set}$ with the same cardinality are isomorphic.
	\end{enumerate}
	
\end{example}


Category theory typically doesn't care about the distinction between isomorphic objects.




\begin{df}[Epi--mono factorization]
	An arrow $f \colon C \to D$ has an \textit{epi--mono factorization} iff there is an epic arrow $e \colon B \epic C$ and a monic arrow $m \colon C \monic D$ s.t. $f = m \circ e$.
\end{df}


\begin{center}
\begin{tikzcd}
	B \arrow[rr, "f"] \arrow[rd, "e"', two heads] &  & D \\
	 & C \arrow[ru, "m"', tail] &  
\end{tikzcd}
\end{center}


\begin{thm} 
	In $\mathsf{Set}$:
	\begin{enumerate}[itemsep=5pt,parsep=5pt,leftmargin=3em,topsep=5pt,label=(\arabic*)] %% or label = \alph*, \roman*
		\item 
		Every arrow has an epi--mono factorization.

		\item 
		If $f \colon B \to D$ \textit{factors} both as 
		$B \stackrel{e}{\epic} C \stackrel{m}{\monic} D$ and as 
		$B \stackrel{e'}{\epic} C' \stackrel{m'}{\monic} D$, 
		then $C \cong C'$.
	\end{enumerate}
\begin{center}
\begin{tikzcd}
	& C' \arrow[rd, "m'", tail] &   \\
   B \arrow[rr, "f" description] \arrow[rd, "e"', two heads] \arrow[ru, "e'", two heads] &     & D \\
	& C \arrow[ru, "m"', tail]  &  
\end{tikzcd}
\end{center}
\end{thm}



Note that the epi--mono factorization is not always available in a given category. 



\vspace{3em}

\textit{groupoid} \zh{广群}




%%=============================================================================
\subsection{Initial and final objects}


Characterizing an object by the way it relates to other objects. 
% 
For instance, 
in $\mathsf{Set}$, 
the \textit{empty set} is distinguished by being such that there is one and only one arrow from it to any object. 
And a \textit{singleton} is distinguished by being such that there is one and only one arrow to it from any object.




\begin{df}[Initial and final objects]
	An object $X$ is \textit{initial} in a category $\mathsf{C}$ if for every object $A$ in $\mathsf{C}$ there is a unique arrow $\alpha! \colon X \to A$, 
	and \textit{final}/\textit{terminal} if for all $A$ there is a unique $\alpha! \colon A \to X$.
\end{df}





\begin{remark}
	The use of `!' to signal the unique arrows from an initial object or to a terminal object is quite common. 

	The initial and finial objects in a given category may not unique.	

	A category may have zero, one or many initial objects, 
	and (independently of that) may have zero, one or many terminal objects.

	An object can be both initial and terminal.
\end{remark}



\begin{example}
	\begin{enumerate}[itemsep=5pt,parsep=5pt,leftmargin=3em,topsep=5pt,label=(\alph*)] %% or label = \alph*, \roman*
		\item 
		In $\mathsf{Set}$, 
		the empty set $\emptyset$ is the {\color{teal} unique initial object}, 
		and the final objects are precisely the singletons. 

		\item 
		In $\mathsf{Set}_\star$, 
		the category whose objects are non--empty sets equipped with a distinguished element and whose arrows are functions preserving distinguished element,
		each singleton is both initial and terminal.

		\item 
		In $\mathsf{Rel}$, 
		the category of sets and relations, 
		the empty set is both the sole initial and sole terminal object.


		\item 
		In $\mathsf{Grp}$, 
		the one--element group is an initial object, 
		also final.


		\item 
		In $\mathsf{Prop}_L$, 
		the category of propositional in FO language $L$, 
		$\bot$ is initial and $\top$ is terminal.
	\end{enumerate}
\end{example}




\begin{df}[Null objects]
	An object $O$ in a category $\mathsf{C}$ is a \textit{null/zero object} \zh{零对象} iff it is both initial and final.	
\end{df}



For every general result about initial objects, 
there is a dual result about terminal objects.



\begin{thm}[Uniqueness up to unique isomorphism]
	If initial objects exist, 
	then they are `\textit{unique up to unique isomorphism}', 
	that is, 
	if the $\mathsf{C}$--objects $I$ and $J$ are both initial, 
	then there is a unique isomorphism $f \colon I \isoarrow J$ in $\mathsf{C}$. 
	% 
	Dually for terminal objects.
\end{thm}




\begin{thm}
	If $I$ is initial in $\mathsf{C}$ and $I \cong J$, 
	then $J$ is also initial. 
	% 
	Dually for terminal objects.
\end{thm}


\begin{df}
	We now use `{\color{purple} 0}' to denote an initial object (assuming it exists)  and `{\color{purple} 1}' to denote a terminal object.
\end{df}



\begin{prop}
	In a category with a terminal object, 
	any arrow $f \colon 1 \to X$ is \textit{monic}.
\end{prop}





%%=============================================================================
\subsection{Products \& Coproducts}



%%-------------------------------------
\subsubsection{Warm up}



%%-------------------------------------
\subsubsection{}


\begin{df}[Products \& Coproducts]
	A \textit{product} of two objects $A_0$ and $A_1$ in a category $\mathsf{C}$ consists of a triple 
	$(A, A \xrightarrow{\alpha_0} A_0, A \xrightarrow{\alpha_1} A_1)$, 
	such that for every triple 
	$(A', A' \xrightarrow{\alpha'_0} A_0, A' \xrightarrow{\alpha'_1} A_1)$ there is a unique arrow $f \colon A' \to A$ such that $\alpha_i \circ f = \alpha'_i$ for both $i$.


	% figure
	\begin{figure}[h]
		\centering
		\begin{tikzcd}
			& A \arrow[ld, "\alpha_0"'] \arrow[rd, "\alpha_1"]   &     \\
		A_0 &    & A_1 \\
			& A' \arrow[uu, "f" description] \arrow[ru, "\alpha'_1"'] \arrow[lu, "\alpha'_0"] &    
		\end{tikzcd}
		\caption{product of objects}
		\label{fig: product-of-objects}
	\end{figure}
	
	\textbf{Coproducts} of $A_0$ and $A_1$ are defined dually as triples $(A, A_0 \xrightarrow{\alpha_0} A, A_1 \xrightarrow{\alpha_1} A)$, 
	such that for every triple 
	$(A', A_0 \xrightarrow{\alpha'_0} A', A_1 \xrightarrow{\alpha'_1} A')$ there is a unique arrow $f \colon A \to A'$ such that $f \circ \alpha_i  = \alpha'_i$ for each $i$. 
	% 
	% figure
	\begin{figure}[h]
		\centering
		\begin{tikzcd}
			& A \arrow[dd, "f" description] &      \\
			A_0 \arrow[ru, "\alpha_0"] \arrow[rd, "\alpha'_0"'] &                               & A_1 \arrow[lu, "\alpha_1"'] \arrow[ld, "\alpha'_1"] \\
			& A'    &           
		\end{tikzcd}
		\caption{Coproducts}
		\label{fig: coproducts}
	\end{figure}
\end{df}



\begin{example}
	The category $\mathsf{Set}$ has both products and coproducts --- 
	that is, 
	every pair $(S_0,S_1)$ of sets has both a product 
	--- 
	the cartesian product $S_0 \times S_1$ together with the two projection functions $\pi_i \colon S_0 \times S_1 \to S_i$. 


	% figure
	\begin{center}
	\begin{tikzcd}
		& S_0 \times S_1 \arrow[ld, "\pi_0"'] \arrow[rd, "\pi_1"] &     \\
	S_0 &                                                         & S_1
	\end{tikzcd}
	\end{center}


	And a coproduct ---
	the disjoint union $S_0 \uplus  S_1 = S_0 \times \{0\} \cup S_1 \times \{1\}$ together with the coproduct maps $\kappa_0$ and $\kappa_1$ given by 
	$\kappa_i(s) = (s,i)$.


	% figure
	\begin{center}
	\begin{tikzcd}
		& S_0 \uplus S_1 &        \\
		S_0 \arrow[ru, "\kappa_0"] &    & S_1 \arrow[lu, "\kappa_1"']
	\end{tikzcd}
	\end{center}
\end{example}



%%=============================================================================
\clearpage

\section{Coalgebra}


Coalgebra can be conceived as a general and uniform theory of \textit{dynamic systems}, taken in a broad sense. 
% 
Many structures in mathematics and theoretical computer science can naturally be represented as coalgebras.



Probably the first example was provided by Aczel [2], 
who models \textit{transition systems} and \textit{non--well--founded sets} as coalgebras. 


\vspace{2em}


For \textsf{modal logicians}, 
it will be Kripke \textsf{frames} and \textsf{models} that provide the prime examples of coalgebras, 
this link goes back to at least Abramsky [1].


In fact, 
the modal model theory is coalgebraic in nature, so modal logicians entering the field will have much the same experience as group theorists learning about universal algebra, 
in that they will recognize many familiar notions and results, 
lifted to a higher level of generality and abstraction.

\vspace{2em}


- {\color{purple} algebraic operations} are ways to construct complex objects out of simple ones, 
{\color{purple} coalgebraic operations}, going out of the carrier set, should be seen as ways to unfold or observe objects.



- coalgebras over the base category $\mathsf{C}$ are dual to algebras over the opposite category $\mathsf{C}^{op}$. 
This explains not only the name ``coalgebra''.


- the link between modal logics and coalgebra is so tight, that one may even claim that {\color{teal} modal logic is the natural logic for coalgebras --- just like equational logic is that for algebra}.





\begin{df}[Coalgebras] \zh{余代数}
	Given an endofunctor (see Def.~\ref{}) $\Omega$ on a category $\mathsf{C}$, 
	an \textbf{$\Omega$--coalgebra} is a pair {\color{red} $\mathbb{A} = (A,\alpha)$}, 
	also denoted as {\color{red} $\mathbb{A} = (A, \alpha \colon A \to \Omega A)$} or {\color{red} $\mathbb{A} = (A, A \xrightarrow{\alpha} \Omega A)$}, 
	where $A$ is an object of $\mathsf{C}$ called the \textbf{carrier} of $\mathbb{A}$, 
	and $\alpha \colon A \to \Omega A$ is an arrow in $\mathsf{C}$, 
	called the \textbf{transition map} of $\mathbb{A}$. 

	In that case that $\Omega$ is an endofunctor on $\mathsf{Set}$, 
	$\Omega$--coalgebras may also be called \textbf{$\Omega$--systems}. 
	% 
	A \textbf{pointed $\Omega$-system} is a triple {\color{red} $(A,\alpha,a)$} where $(A,\alpha)$ is an $\Omega$-system and $a \in A$. 
\end{df}


An $\Omega$--coalgebra $\mathbb{A} = (A,\sigma)$ can be pictured by
% figure
\begin{figure}[h]
	\centering
	\begin{tikzcd}
		A \arrow[r, "\alpha"] & \Omega A
	\end{tikzcd}
	\caption{$\Omega$--Coalgebra $\mathbb{A} = (A,\sigma)$}
	\label{fig:}
\end{figure}



\begin{example}[Automata]
	\textsf{Deterministic automata} are usually modeled as quintuples 
	\[
		\mathbb{A} = 
		(Q,a_0,\Sigma,\delta,F)
	\]
	such that $Q$ is the state space of the automaton, 
	$a_0 \in Q$ is its \textit{initial state}, 
	$\Sigma$ its \textit{alphabet}, 
	$\delta \colon Q \times \Sigma \to Q$ its \textit{transition function} and finally, 
	$F \subseteq Q$ its \textit{accepting states}. 


	We can represent $F$ by by its characteristic map $C_F \colon Q \to 2$ (where $2$ denoting the set $\{0,1\}$) such that $C_F (a) = 1$ if $a \in F$ and $C_F (a) = 0$ otherwise. 
	% 
	Furthermore, we can view $\delta$ as a map from $Q \to Q^\Sigma$:
	\[
		\delta \colon Q \times \Sigma \to Q 
			\qquad \leadsto \qquad 
		\delta \colon Q \to (\Sigma \to Q) 
			\qquad \leadsto \qquad 
		\delta \colon Q \to Q^\Sigma
	\]
	where $Q^\Sigma$ denotes the collection of maps from $\Sigma$ to $Q$. 

	Thus, 
	we may represent a deterministic automaton over the alphabet $\Sigma$ as a {\color{purple} \textit{pointed system}} over the functor $\Omega \colon S \mapsto 2 \times S^\Sigma$ for any set $S$.
\end{example}




\begin{example}[Kripke Frames \& Models]
	We now see that Kripke frames and models are in fact coalgebras in disguise.	

	\begin{enumerate}[itemsep=5pt,parsep=5pt,leftmargin=3em,topsep=5pt,label=(\arabic*)]
	
	\item 
	\textbf{Frame}: 
	Considering the frame $\mathfrak{F} = (W,R)$ (for the basic modal similarity type).
	% 
	The crucial observation is that the binary relation $R$ on $W$ can be represented as the function 
	\[
		R[\cdot] \colon W \to \mathcal{P}W
	\]
	mapping a point $w$ to its $R$--successors $R[w] = \{u \in W \mid Rwu\}$. 
	% 
	Thus frames $\mathfrak{F} = (W,R)$ correspond to coalgebras over the {\color{purple} powerset functor $\mathcal{P}$}. 
	% 
	Therefore, 
	a frame $\mathfrak{F} = (W,R)$ is a {\color{purple} $\mathcal{P}$--coalgebra} or {\color{purple} $\mathcal{P}$--system} $(W, R \colon W \to \mathcal{P}W)$. 
	% 
	\textbf{Pointed frame} $(\mathfrak{F},w)$ is just the {\color{purple} pointed $\mathcal{P}$--system} $(W, W \xrightarrow{R} \mathcal{P}W, w)$.

	(Note that:
	the powerset functor $\mathcal{P}$ maps any set $S$ to its powerset $\mathcal{P}(S)$ and a function $f \colon S \to S'$ to the image map $\mathcal{P}f$ given by 
	$(\mathcal{P}f)(X) \coloneqq f[X] = \{f(x) \mid x \in X\}$.)

	

	\item 
	\textbf{Image finite frames},
	that is, frames in which $R[w]$ is finite for all points $w$, 
	correspond to coalgebras over the {\color{purple} finitary powerset functor $\mathcal{P}_\omega$}.

	
	\item  
	\textbf{Ternary frames}: 
	% 
	$W$ with a ternary relation $T \subseteq W^3$ forms a ternary frame $(W,T)$ (a frame for temporal logic with since $S$ or until $U$). 
	% 
	Similarly, $T$ can be represented as 
	\[
		T[\cdot] \colon W \to \mathcal{P}(W^2)
	\]
	s.t. $T[w] = \{(w_1,w_2) \in W^2 \mid (w,w_1,w_2) \in T\}$. 
	% 
	Thus, 
	a ternary frame $(W,T)$ is a coalgebra $(W, T \colon W \to \mathcal{P}(W^2))$  under the functor $\Omega$ which $\Omega \colon S \mapsto \mathcal{P}(S^2)$ for any set $S$, 
	and for function $f \colon S \to S'$,
	$\Omega f$  is given by 
	$(\Omega f)(R) \coloneqq \{(f(x_1),f(x_2) \mid (x_1,x_2) \in R)\}$ where $R$ is a (binary) relation in $\mathcal{P}(S^2)$.

	

	\item \textbf{Models}: 
	Now concerning models with the form $\mathfrak{M} = (W,R,V)$. 
	% 
	It is easy to see that a valuation $V \colon \mathsf{PROP} \to \mathcal{P}(W)$ could equivalently have been defined as a $\mathcal{P}(\mathsf{PROP})$--coloring of $W$, 
	that is, 
	mapping a state $w$ to the collection 
	\[
		V^{-1}[s] = \{p \in \mathsf{PROP} \mid s \in V(p)\}
	\]
	of proposition letters holding at $w$. 
	% 
	Thus models can be identified with coalgebras of the functor $\Omega$ given by 
	\[
		\Omega \colon W \mapsto \mathcal{P}(\mathsf{PROP}) \times \mathcal{P}(W)
	\]
	for any set $W$,  
	and for function $f \colon S \to S'$, 
	$\Omega f$ is given by $(\Omega f)(X',X'') \coloneqq (f[X],f[X''])$ where $X \subseteq S$ and $X'' \subseteq S''$.

	\item Recap:
	
	For frame $\mathfrak{F} = (W,R)$ and model $\mathfrak{M} = (W,R,V)$, 
	the corresponding coalgebras are \\
	
	$\begin{array}{lll}
		\mathfrak{F} = (W,R): 
		& 
		(W, W \xrightarrow{R} \mathcal{P}W)
		& 
		\text{where~}
		R(w) =\{u \in W \mid Rwu\}  \\  

		\\

		\mathfrak{M} = (W,R,V):
		&
		(W, W \xrightarrow{\alpha} \mathcal{P}(\mathsf{PROP}) \times \mathcal{P}W) 
		&
		\text{where~}
		\alpha(w)=\langle \{p \mid w \Vdash p\},R(w)  \rangle
	\end{array}$
	\end{enumerate}
\end{example}



\begin{example}[Neighborhood frames]
	\textbf{contravariant powerset functor} $\breve{\mathcal{P}}$
\end{example}




% /////////////////////////////////////////////////////////////////////////////
\clearpage
% \bibliographystyle{plain}

% \bibliography{/Users/chenxin/Desktop/chenxin.bib}
\printbibliography
\addcontentsline{toc}{section}{References}
\end{document}