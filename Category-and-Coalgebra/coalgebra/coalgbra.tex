\section{Coalgebra}


Coalgebra can be conceived as a general and uniform theory of \textit{dynamic systems}, taken in a broad sense. 
% 
Many structures in mathematics and theoretical computer science can naturally be represented as coalgebras.



Probably the first example was provided by Aczel [2], 
who models \textit{transition systems} and \textit{non--well--founded sets} as coalgebras. 


\vspace{2em}


For \textsf{modal logicians}, 
it will be Kripke \textsf{frames} and \textsf{models} that provide the prime examples of coalgebras, 
this link goes back to at least Abramsky [1].


In fact, 
the modal model theory is coalgebraic in nature, so modal logicians entering the field will have much the same experience as group theorists learning about universal algebra, 
in that they will recognize many familiar notions and results, 
lifted to a higher level of generality and abstraction.

\vspace{2em}


- {\color{purple} algebraic operations} are ways to construct complex objects out of simple ones, 
{\color{purple} coalgebraic operations}, going out of the carrier set, should be seen as ways to unfold or observe objects.



- coalgebras over the base category $\mathsf{C}$ are dual to algebras over the opposite category $\mathsf{C}^{op}$. 
This explains not only the name ``coalgebra''.


- the link between modal logics and coalgebra is so tight, that one may even claim that {\color{teal} modal logic is the natural logic for coalgebras --- just like equational logic is that for algebra}.





\begin{df}[Coalgebras] \zh{余代数}
	Given an endofunctor (see Def.~\ref{}) $\Omega$ on a category $\mathsf{C}$, 
	an \textbf{$\Omega$--coalgebra} is a pair {\color{red} $\mathbb{A} = (A,\alpha)$}, 
	also denoted as {\color{red} $\mathbb{A} = (A, \alpha \colon A \to \Omega A)$} or {\color{red} $\mathbb{A} = (A, A \xrightarrow{\alpha} \Omega A)$}, 
	where $A$ is an object of $\mathsf{C}$ called the \textbf{carrier} of $\mathbb{A}$, 
	and $\alpha \colon A \to \Omega A$ is an arrow in $\mathsf{C}$, 
	called the \textbf{transition map} of $\mathbb{A}$. 

	In that case that $\Omega$ is an endofunctor on $\mathsf{Set}$, 
	$\Omega$--coalgebras may also be called \textbf{$\Omega$--systems}. 
	% 
	A \textbf{pointed $\Omega$-system} is a triple {\color{red} $(A,\alpha,a)$} where $(A,\alpha)$ is an $\Omega$-system and $a \in A$. 
\end{df}


An $\Omega$--coalgebra $\mathbb{A} = (A,\sigma)$ can be pictured by
% figure
\begin{figure}[h]
	\centering
	\begin{tikzcd}
		A \arrow[r, "\alpha"] & \Omega A
	\end{tikzcd}
	\caption{$\Omega$--Coalgebra $\mathbb{A} = (A,\sigma)$}
	\label{fig:}
\end{figure}



\begin{example}[Automata]
	\textsf{Deterministic automata} are usually modeled as quintuples 
	\[
		\mathbb{A} = 
		(Q,a_0,\Sigma,\delta,F)
	\]
	such that $Q$ is the state space of the automaton, 
	$a_0 \in Q$ is its \textit{initial state}, 
	$\Sigma$ its \textit{alphabet}, 
	$\delta \colon Q \times \Sigma \to Q$ its \textit{transition function} and finally, 
	$F \subseteq Q$ its \textit{accepting states}. 


	We can represent $F$ by by its characteristic map $C_F \colon Q \to 2$ (where $2$ denoting the set $\{0,1\}$) such that $C_F (a) = 1$ if $a \in F$ and $C_F (a) = 0$ otherwise. 
	% 
	Furthermore, we can view $\delta$ as a map from $Q \to Q^\Sigma$:
	\[
		\delta \colon Q \times \Sigma \to Q 
			\qquad \leadsto \qquad 
		\delta \colon Q \to (\Sigma \to Q) 
			\qquad \leadsto \qquad 
		\delta \colon Q \to Q^\Sigma
	\]
	where $Q^\Sigma$ denotes the collection of maps from $\Sigma$ to $Q$. 

	Thus, 
	we may represent a deterministic automaton over the alphabet $\Sigma$ as a {\color{purple} \textit{pointed system}} over the functor $\Omega \colon S \mapsto 2 \times S^\Sigma$ for any set $S$.
\end{example}




\begin{example}[Kripke Frames \& Models]
	We now see that Kripke frames and models are in fact coalgebras in disguise.	

	\begin{enumerate}[itemsep=5pt,parsep=5pt,leftmargin=3em,topsep=5pt,label=(\arabic*)]
	
	\item 
	\textbf{Frame}: 
	Considering the frame $\mathfrak{F} = (W,R)$ (for the basic modal similarity type).
	% 
	The crucial observation is that the binary relation $R$ on $W$ can be represented as the function 
	\[
		R[\cdot] \colon W \to \mathcal{P}W
	\]
	mapping a point $w$ to its $R$--successors $R[w] = \{u \in W \mid Rwu\}$. 
	% 
	Thus frames $\mathfrak{F} = (W,R)$ correspond to coalgebras over the {\color{purple} powerset functor $\mathcal{P}$}. 
	% 
	Therefore, 
	a frame $\mathfrak{F} = (W,R)$ is a {\color{purple} $\mathcal{P}$--coalgebra} or {\color{purple} $\mathcal{P}$--system} $(W, R \colon W \to \mathcal{P}W)$. 
	% 
	\textbf{Pointed frame} $(\mathfrak{F},w)$ is just the {\color{purple} pointed $\mathcal{P}$--system} $(W, W \xrightarrow{R} \mathcal{P}W, w)$.

	(Note that:
	the powerset functor $\mathcal{P}$ maps any set $S$ to its powerset $\mathcal{P}(S)$ and a function $f \colon S \to S'$ to the image map $\mathcal{P}f$ given by 
	$(\mathcal{P}f)(X) \coloneqq f[X] = \{f(x) \mid x \in X\}$.)

	

	\item 
	\textbf{Image finite frames},
	that is, frames in which $R[w]$ is finite for all points $w$, 
	correspond to coalgebras over the {\color{purple} finitary powerset functor $\mathcal{P}_\omega$}.

	
	\item  
	\textbf{Ternary frames}: 
	% 
	$W$ with a ternary relation $T \subseteq W^3$ forms a ternary frame $(W,T)$ (a frame for temporal logic with since $S$ or until $U$). 
	% 
	Similarly, $T$ can be represented as 
	\[
		T[\cdot] \colon W \to \mathcal{P}(W^2)
	\]
	s.t. $T[w] = \{(w_1,w_2) \in W^2 \mid (w,w_1,w_2) \in T\}$. 
	% 
	Thus, 
	a ternary frame $(W,T)$ is a coalgebra $(W, T \colon W \to \mathcal{P}(W^2))$  under the functor $\Omega$ which $\Omega \colon S \mapsto \mathcal{P}(S^2)$ for any set $S$, 
	and for function $f \colon S \to S'$,
	$\Omega f$  is given by 
	$(\Omega f)(R) \coloneqq \{(f(x_1),f(x_2) \mid (x_1,x_2) \in R)\}$ where $R$ is a (binary) relation in $\mathcal{P}(S^2)$.

	

	\item \textbf{Models}: 
	Now concerning models with the form $\mathfrak{M} = (W,R,V)$. 
	% 
	It is easy to see that a valuation $V \colon \mathsf{PROP} \to \mathcal{P}(W)$ could equivalently have been defined as a $\mathcal{P}(\mathsf{PROP})$--coloring of $W$, 
	that is, 
	mapping a state $w$ to the collection 
	\[
		V^{-1}[s] = \{p \in \mathsf{PROP} \mid s \in V(p)\}
	\]
	of proposition letters holding at $w$. 
	% 
	Thus models can be identified with coalgebras of the functor $\Omega$ given by 
	\[
		\Omega \colon W \mapsto \mathcal{P}(\mathsf{PROP}) \times \mathcal{P}(W)
	\]
	for any set $W$,  
	and for function $f \colon S \to S'$, 
	$\Omega f$ is given by $(\Omega f)(X',X'') \coloneqq (f[X],f[X''])$ where $X \subseteq S$ and $X'' \subseteq S''$.

	\item Recap:
	
	For frame $\mathfrak{F} = (W,R)$ and model $\mathfrak{M} = (W,R,V)$, 
	the corresponding coalgebras are \\
	
	$\begin{array}{lll}
		\mathfrak{F} = (W,R): 
		& 
		(W, W \xrightarrow{R} \mathcal{P}W)
		& 
		\text{where~}
		R(w) =\{u \in W \mid Rwu\}  \\  

		\\

		\mathfrak{M} = (W,R,V):
		&
		(W, W \xrightarrow{\alpha} \mathcal{P}(\mathsf{PROP}) \times \mathcal{P}W) 
		&
		\text{where~}
		\alpha(w)=\langle \{p \mid w \Vdash p\},R(w)  \rangle
	\end{array}$
	\end{enumerate}
\end{example}



\begin{example}[Neighborhood frames]
	\textbf{contravariant powerset functor} $\breve{\mathcal{P}}$
\end{example}
