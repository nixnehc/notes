\section{Functors}



\begin{df}[Functors]
	A \textbf{functor} $\Omega \colon \mathsf{C} \to \mathsf{D}$ from category $\mathsf{C}$ to category $\mathsf{D}$ consists of an operation mapping objects and arrows of $\mathsf{C}$ to objects and arrows of $\mathsf{D}$, 
	respectively, 
	in such a way that for all objects and arrows involved
    % 
	\begin{enumerate}[itemsep=5pt,parsep=5pt,leftmargin=4em,topsep=5pt,label=(\arabic*)]
		\item 
        $\Omega f \colon \Omega A \to \Omega B$ if $f \colon A \to B$.
		
		\item 
        $\Omega(Id_A) = Id_{\Omega A}$. 

		\item 
        $\Omega(g \circ f) = (\Omega g) \circ (\Omega f)$
	\end{enumerate}

	A functor $\Omega \colon \mathsf{C} \to \mathsf{D}^{op}$ is sometimes called a \textbf{contravariant functor} from $\mathsf{C}$ to $\mathsf{D}$.


	An \textbf{endofunctor} on $\mathsf{C}$ is a functor $\Omega \colon \mathsf{C} \to \mathsf{C}$.
\end{df}




\begin{example}[Set functors]
	We consider the following endofunctor on $\mathsf{Set}$. 
	\begin{enumerate}[itemsep=5pt,parsep=5pt,leftmargin=3em,topsep=5pt,label=(\arabic*)] %% or label = \alph*, \roman*
		\item For a fixed set $C$, 
		the \textbf{constant functor} mapping all sets to $C$ and all arrows to $id_C$; 
		this functor is denoted as {\color{red} $C$}.


		\item The \textbf{power set functor} $\mathcal{P}$, 
		which maps amy set $S$ to its powerset $\mathcal{P} S$, 
		and any map $f \colon S \to S'$ to the map 
		$\mathcal{P} f \colon \mathcal{P} S \to \mathcal{P} S'$ given by $\mathcal{P} f \colon X \mapsto f[X]$, 
		where $f[X] = \{fx \in S' \mid x \in X \subseteq S\}$.


		\item For every cardinal $\kappa$, 
		the variant {\color{red} $\mathcal{P}_\kappa$} of the powerset functor, 
		which maps any set $S$ to the collection $\mathcal{P}_\kappa \coloneqq \{X \subseteq S \mid |X| < \kappa\} \subseteq \mathcal{P}S$, 
		and agrees with $\mathcal{P}$ on the arrows for which is defined.
	\end{enumerate}	
\end{example}




\begin{df}[Product of functors]
	Given two functors $\Omega_0$ and $\Omega_1$, 
	their \textbf{product functor} $\Omega_0 \times \Omega_1$ is given on objects by 
	\[
		(\Omega_0 \times \Omega_1) S 
		\coloneqq 
		\Omega_0 S \times \Omega_1 S,
	\]
	while for $f \colon S \to S'$ 
	the map $(\Omega_0 \times \Omega_1)f$ is given as 
	\[
		((\Omega_0 \times \Omega_1)f) (\sigma_0,\sigma_1) 
		\coloneqq 
		((\Omega_0 f)(\sigma_0), (\Omega_1 f)(\sigma_1)).
	\]

	The \textbf{coproduct functor} is defined similarly.
\end{df}




\begin{df}[Identity functor]
	Every category $\mathsf{C}$ admits the \textbf{identity functor} $\mathcal{I}_\mathsf{C} \colon \mathsf{C} \to \mathsf{C}$ which is the identity on both objects and arrows of $\mathsf{C}$.
\end{df}






\begin{df}[Natural transformation]
	Let $\mathsf{C}$ and $\mathsf{D}$ be two categories, and let $\Omega$ and $\Psi$ be two functors from $\mathsf{C}$ to $\mathsf{D}$. 
	A \textbf{natural transformation} $\tau$ from $\Omega$ to $\Psi$, 
	notation {\color{red} $\tau \colon \Omega \Rightarrow \Psi$}, 
	consists of $\mathsf{D}$-arrows $\tau_A \colon \Omega A \to \Psi A$ such that 
	\[
		\tau_B \circ \Omega f = \Psi f \circ \tau_A
	\]
	for each $f \colon A \to B$ in $\mathsf{C}$.

	% figure
	\begin{figure}[h]
		\centering
		\begin{tikzcd}
			& A \arrow[d, "f"'] & {} \arrow[r, "\Omega", dashed, bend left] & {} & \Omega A \arrow[r, "\Omega f"] \arrow[d, "\tau_A"'] \arrow[rd] & \Omega B \arrow[d, "\tau_B"] &            \\
 			\mathsf{C} & B                 & {} \arrow[r, "\Psi"', dashed, bend right] & {} & \Psi A \arrow[r, "\Psi f"']                                    & \Psi B                       & \mathsf{D}
 		\end{tikzcd}
		\caption{$\tau \colon \Omega \Rightarrow \Psi$}
		\label{fig:}
	\end{figure}

\end{df}




???? equivalent , isomorphic,   dual/ dually equivalent






% /////////////////////////////////////////////////////////////////////////////
\clearpage
